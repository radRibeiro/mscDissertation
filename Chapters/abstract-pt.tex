%!TEX root = ../template.tex
%%%%%%%%%%%%%%%%%%%%%%%%%%%%%%%%%%%%%%%%%%%%%%%%%%%%%%%%%%%%%%%%%%%%
%% abstrac-pt.tex
%% NOVA thesis document file
%%
%% Abstract in Portuguese
%%%%%%%%%%%%%%%%%%%%%%%%%%%%%%%%%%%%%%%%%%%%%%%%%%%%%%%%%%%%%%%%%%%%



%Independentemente da língua em que está escrita a dissertação, é necessário um resumo na língua do texto principal e um resumo noutra língua.  Assume-se que as duas línguas em questão serão sempre o Português e o Inglês.
%
%O \emph{template} colocará automaticamente em primeiro lugar o resumo na língua do texto principal e depois o resumo na outra língua.  Por exemplo, se a dissertação está escrita em Português, primeiro aparecerá o resumo em Português, depois em Inglês, seguido do texto principal em Português. Se a dissertação está escrita em Inglês, primeiro aparecerá o resumo em Inglês, depois em Português, seguido do texto principal em Inglês.
%
%O resumo não deve exceder uma página e deve responder às seguintes questões:
%\begin{itemize}
%% What's the problem?
%	\item Qual é o problema?
As proteinas são um dos nutrientes indispensáveis à vida na Terra e a um longo prazo no Universo em geral, a capacidade dos seus compostos (que segundo o ramo da biologia/bioquimica se ditam por amino-ácidos) de se juntarem e formarem vastos complexos permite o crescimento fisico dos seres-vivos. 
Contudo, este processo de ajuntamento em termos informáticos apenas pode ser determinado por matrizes tri-dimensionais em situações em que temos vários compostos interligados, já que os mesmo assumem estruturas tri-dimensionais, sendo mais dificil determinar experimentalmente uma previsão correta sobre estes ultimos do que numa situação em que temos proteinas singulares pois existem muitas combinações possiveis para a docagem entre os pares, logo é necessário recorrer a ferramentas que oferecem a computação adequada para determinar estas previsões, a fim de poder-se avançar no estudo das interações entre proteinas e na concepção de produtos à base das mesmas.\\
%% Why is it interesting?
%	\item Porque é que ele é interessante?
%% What's the solution?
%	\item Qual é a solução?
O presente documento aborda uma proposição para a paralelização do algoritmo BiGGER implementado pelo prof. Ludwig Krippahl e outros professores associados à criação do algoritmo, recorrendo a técnicas de computação acelerada i.e. utilizar o GPU da máquina em que corre o algoritmo para auxiliar o CPU na computação que é necessária para o algoritmo na versão actual, de forma a que com mais recursos à disposição, o tempo de execução do BiGGER baixe drasticamente por consequencia do aumento significativo de perfomance.\\
Em caso de sucesso, a complexidade futura do algoritmo permitirá a adição de mais vantagens face aos seus concorrentes, e por consequência, uma proposta de valor para quem pretenda utilizar o open-chemera como ferramenta de trabalho eficiente no estudo das interações entre os complexos de proteinas em qualquer máquina que tenha uma placa grafica com as caracteristicas adequadas.
%% What follows from the solution?
%	\item O que resulta (implicações) da solução?
%\end{itemize}
%
%E agora vamos fazer um teste com uma quebra de linha no hífen a ver se a \LaTeX\ duplica o hífen na linha seguinte…
%
%zzzz zzz zzzz zzz zzzz zzz zzzz zzz zzzz zzz zzzz zzz zzzz zzz zzzz zzz zzzz comentar"-lhe zzz zzzz zzz zzzz 
%
%Sim!  Funciona! :)


% Palavras-chave do resumo em Português
\begin{keywords}
proteinas, docagem de proteinas,computação acelerada,GPU, Bio-Informática, BiGGER
\end{keywords}
% to add an extra black line

%!TEX root = ../template.tex
%%%%%%%%%%%%%%%%%%%%%%%%%%%%%%%%%%%%%%%%%%%%%%%%%%%%%%%%%%%%%%%%%%%%
%% abstrac-pt.tex
%% NOVA thesis document file
%%
%% Abstract in Portuguese
%%%%%%%%%%%%%%%%%%%%%%%%%%%%%%%%%%%%%%%%%%%%%%%%%%%%%%%%%%%%%%%%%%%%



%Independentemente da língua em que está escrita a dissertação, é necessário um resumo na língua do texto principal e um resumo noutra língua.  Assume-se que as duas línguas em questão serão sempre o Português e o Inglês.
%
%O \emph{template} colocará automaticamente em primeiro lugar o resumo na língua do texto principal e depois o resumo na outra língua.  Por exemplo, se a dissertação está escrita em Português, primeiro aparecerá o resumo em Português, depois em Inglês, seguido do texto principal em Português. Se a dissertação está escrita em Inglês, primeiro aparecerá o resumo em Inglês, depois em Português, seguido do texto principal em Inglês.
%
%O resumo não deve exceder uma página e deve responder às seguintes questões:
%\begin{itemize}
%% What's the problem?
%	\item Qual é o problema?
Na área cientifica da Bio-Informática, o estudo das interações entre as proteinas tem sido objeto de interesse. 
No entanto conseguir determinar computacionalmente e com precisão como é que as mesmas se unem é dificil. 
Existem, no entanto diversos métodos e algoritmos para prever o ajuntamento das proteinas. 
Um desses métodos é o BiGGER, criado pelo prof. Ludwig Krippahl, prof. Nuno Palma e outros integrados no projeto. 
Este algoritmo assume caracteristicas que lhe dão uma complexidade temporal inferior aos demais, pelo que os tempos de execução do BiGGER são menores do que a maior parte dos algoritmos e respetivos programas de docking.
%% Why is it interesting?
%	\item Porque é que ele é interessante?
%Conseguir perceber como é que as proteinas interagem implica avanços na concepção e desenho de drogas assistido por computador 
%% What's the solution?

O presente documento aborda uma proposição para a paralelização do algoritmo BiGGER.
%	\item Qual é a solução?
 A implementação será feita recorrendo a técnicas de computação acelerada i.e. utilizar o GPU da máquina em que corre o algoritmo para auxiliar o CPU na computação que é necessária. 
 
 Tendo mais recursos à disposição, é esperado que o tempo de execução do BiGGER baixe drasticamente por consequencia do aumento significativo de perfomance face à versão sequencial.
%% What follows from the solution?
Em caso de sucesso, a complexidade futura do algoritmo permitirá a adição de mais vantagens face aos seus concorrentes.
% O que resulta (implicações) da solução?
Por consequencia deste aumento de performance, uma proposta de valor para quem pretenda utilizar o open-chemera será ter uma ferramenta de trabalho eficiente no estudo das interações entre as proteinas em qualquer máquina que tenha uma placa gráfica com as caracteristicas adequadas.
%\end{itemize}
%
%E agora vamos fazer um teste com uma quebra de linha no hífen a ver se a \LaTeX\ duplica o hífen na linha seguinte…
%
%zzzz zzz zzzz zzz zzzz zzz zzzz zzz zzzz zzz zzzz zzz zzzz zzz zzzz zzz zzzz comentar"-lhe zzz zzzz zzz zzzz 
%
%Sim!  Funciona! :)


% Palavras-chave do resumo em Português
\begin{keywords}
proteinas, docagem de proteinas,computação acelerada,GPU, Bio-Informática, BiGGER
\end{keywords}
% to add an extra black line

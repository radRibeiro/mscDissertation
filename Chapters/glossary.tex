%!TEX root = ../template.tex
%%%%%%%%%%%%%%%%%%%%%%%%%%%%%%%%%%%%%%%%%%%%%%%%%%%%%%%%%%%%%%%%%%%%
%% glossary.tex
%% NOVA thesis document file
%%
%% Glossary definition
%%%%%%%%%%%%%%%%%%%%%%%%%%%%%%%%%%%%%%%%%%%%%%%%%%%%%%%%%%%%%%%%%%%%
\newglossaryentry{correlacao} {
	name={correlação}, 
	description={Relação geométrica entre duas posições num plano}
}
\newglossaryentry{complementaridade} {
	name={complementaridade}, 
	description={Principio para descrever como é que duas entidades conseguem se unir}
}
\newglossaryentry{conformacao} {
	name={conformação}, 
	description={Sinónimo de ajustamento}
}
\newglossaryentry{core} {
	name={core}, 
	description={Palavra de origem inglesa para definir a região central de uma entidade}
}
\newglossaryentry{surface} {
	name={surface}, 
	description={Palavra de origem inglesa para definir a região de superficie de uma entidade}
}
\newglossaryentry{assessment} {
	name={assessment}, 
	description={Termo inglês para estudar ou analisar e tomar conclusões sobre uma dada matéria}
}
\newglossaryentry{deprecated} {
	name={deprecated}, 
	description={Da lingua inglesa para caracterizar algo que é usável mas foi considerado como obsoleto, não devendo ser utilizado}
}
\newglossaryentry{bottleneck}{
	name={bottleneck}, 
	description={Zona de código de um dado programa onde é verificado que atrasa a execução deste}
}
\newglossaryentry{cristalografia}{
	name={cristalografia de raios X}, 
	description={Método cientifico usado para elucidação das estruturas das proteínas. A cristalização de proteínas foi descoberta acidentalmente no século XIX (1840), por Hunfeld, antes da descoberta histórica dos raios X por parte de Wilhelm Röntgen (1895). A experiência de Hunfeld envolveu a extração de uma hemoglobina do sangue de uma minhoca, verificando por observação em microscópio  }
}
%where a high proportion of executed instructions occur or where most time is spent during the program's execution (not necessarily the same thing since some instructions are faster than others).
\newglossaryentry{hotspot}{
	name={hotspot}, 
	description={Zona de código de um dado programa onde a proporção de instruções executadas é maior ou onde a fração de tempo de execução é maior}
}
\glsaddall